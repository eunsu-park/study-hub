% Exercises for Lesson 04: Mathematical Typesetting Basics
% Topic: LaTeX
% Solutions to practice problems from the lesson.
% Compile: pdflatex exercises/LaTeX/04_math_basics.tex

\documentclass[12pt, a4paper]{article}
\usepackage[utf8]{inputenc}
\usepackage[T1]{fontenc}
\usepackage{amsmath, amssymb}
\usepackage{array}

\title{Exercises for Lesson 04: Mathematical Typesetting Basics}
\author{Study Project}
\date{}

\begin{document}
\maketitle

% === Exercise 1: Basic Symbols ===
% Problem: Write alpha^2 + beta^2 = gamma^2, x in R and y in C,
% A subset B implies A union B = B.

\section*{Exercise 1: Basic Symbols}

\[
\alpha^2 + \beta^2 = \gamma^2
\]

\[
x \in \mathbb{R}, \quad y \in \mathbb{C}
\]

\[
A \subseteq B \implies A \cup B = B
\]

% === Exercise 2: Fractions and Roots ===
% Problem: (x+1)/(x-1), sqrt(a^2+b^2), cube root of 27, nested fraction.

\section*{Exercise 2: Fractions and Roots}

The fraction $\dfrac{x+1}{x-1}$.

\[
\frac{x+1}{x-1}
\]

The square root of $(a^2 + b^2)$:
\[
\sqrt{a^2 + b^2}
\]

The cube root of 27:
\[
\sqrt[3]{27} = 3
\]

A nested fraction:
\[
\frac{1}{1 + \dfrac{1}{1 + \dfrac{1}{2}}}
= \frac{1}{1 + \dfrac{1}{\dfrac{3}{2}}}
= \frac{1}{1 + \dfrac{2}{3}}
= \frac{1}{\dfrac{5}{3}}
= \frac{3}{5}
\]

% === Exercise 3: Summations and Products ===
% Problem: sum i^2, product (1+1/k), double sum a_{ij}.

\section*{Exercise 3: Summations and Products}

The sum from $i=1$ to $n$ of $i^2$:
\[
\sum_{i=1}^{n} i^2 = \frac{n(n+1)(2n+1)}{6}
\]

The product from $k=1$ to $n$ of $(1 + 1/k)$:
\[
\prod_{k=1}^{n} \left(1 + \frac{1}{k}\right) = n + 1
\]

The double sum:
\[
\sum_{i=1}^{m} \sum_{j=1}^{n} a_{ij}
\]

% === Exercise 4: Integrals ===
% Problem: integral 0 to infinity of e^{-x} dx, double integral, contour integral.

\section*{Exercise 4: Integrals}

\[
\int_{0}^{\infty} e^{-x} \, dx = 1
\]

\[
\iint_{D} f(x,y) \, dA
\]

\[
\oint_{C} z \, dz
\]

% === Exercise 5: Delimiters ===
% Problem: |x|, set {x in R : x^2 < 4}, large fraction in parens,
% evaluated derivative.

\section*{Exercise 5: Delimiters}

Absolute value:
\[
\left| x \right|
\]

Set notation:
\[
\left\{ x \in \mathbb{R} : x^2 < 4 \right\}
\]

Large fraction in parentheses:
\[
\left( \frac{a+b}{c+d} \right)
\]

Evaluated derivative:
\[
\left. \frac{dy}{dx} \right|_{x=0}
\]

% === Exercise 6: Greek Letters ===
% Problem: Table of all lowercase and uppercase Greek letters with commands.

\section*{Exercise 6: Greek Letters}

\begin{center}
\begin{tabular}{lll|lll}
\hline
\textbf{Command} & \textbf{Symbol} & \textbf{Name} &
\textbf{Command} & \textbf{Symbol} & \textbf{Name} \\
\hline
\verb|\alpha|    & $\alpha$    & alpha    & \verb|\nu|       & $\nu$       & nu \\
\verb|\beta|     & $\beta$     & beta     & \verb|\xi|       & $\xi$       & xi \\
\verb|\gamma|    & $\gamma$    & gamma    & \verb|\pi|       & $\pi$       & pi \\
\verb|\delta|    & $\delta$    & delta    & \verb|\rho|      & $\rho$      & rho \\
\verb|\epsilon|  & $\epsilon$  & epsilon  & \verb|\sigma|    & $\sigma$    & sigma \\
\verb|\zeta|     & $\zeta$     & zeta     & \verb|\tau|      & $\tau$      & tau \\
\verb|\eta|      & $\eta$      & eta      & \verb|\upsilon|  & $\upsilon$  & upsilon \\
\verb|\theta|    & $\theta$    & theta    & \verb|\phi|      & $\phi$      & phi \\
\verb|\iota|     & $\iota$     & iota     & \verb|\chi|      & $\chi$      & chi \\
\verb|\kappa|    & $\kappa$    & kappa    & \verb|\psi|      & $\psi$      & psi \\
\verb|\lambda|   & $\lambda$   & lambda   & \verb|\omega|    & $\omega$    & omega \\
\verb|\mu|       & $\mu$       & mu       & & & \\
\hline
\end{tabular}
\end{center}

\begin{center}
\begin{tabular}{lll}
\hline
\textbf{Command} & \textbf{Symbol} & \textbf{Name} \\
\hline
\verb|\Gamma|    & $\Gamma$    & Gamma \\
\verb|\Delta|    & $\Delta$    & Delta \\
\verb|\Theta|    & $\Theta$    & Theta \\
\verb|\Lambda|   & $\Lambda$   & Lambda \\
\verb|\Xi|       & $\Xi$       & Xi \\
\verb|\Pi|       & $\Pi$       & Pi \\
\verb|\Sigma|    & $\Sigma$    & Sigma \\
\verb|\Upsilon|  & $\Upsilon$  & Upsilon \\
\verb|\Phi|      & $\Phi$      & Phi \\
\verb|\Psi|      & $\Psi$      & Psi \\
\verb|\Omega|    & $\Omega$    & Omega \\
\hline
\end{tabular}
\end{center}

% === Exercise 7: Arrows and Relations ===
% Problem: f: A -> B, monotonicity, iff, limit.

\section*{Exercise 7: Arrows and Relations}

Function mapping:
\[
f : A \to B
\]

Monotonicity:
\[
x \leq y \implies f(x) \leq f(y)
\]

Logical equivalence:
\[
A \iff B
\]

Limit:
\[
\lim_{x \to \infty} f(x) = L
\]

% === Exercise 8: Complex Expression ===
% Problem: Cauchy-Schwarz inequality.

\section*{Exercise 8: Cauchy--Schwarz Inequality}

\[
\left| \sum_{i=1}^{n} x_i y_i \right|
\leq
\sqrt{\sum_{i=1}^{n} x_i^2} \; \sqrt{\sum_{i=1}^{n} y_i^2}
\]

% === Exercise 9: Piecewise Function ===
% Problem: Piecewise function using cases.

\section*{Exercise 9: Piecewise Function}

The sign function:
\[
\operatorname{sgn}(x) = \begin{cases}
  +1 & \text{if } x > 0 \\
   0 & \text{if } x = 0 \\
  -1 & \text{if } x < 0
\end{cases}
\]

The ReLU activation function:
\[
\operatorname{ReLU}(x) = \begin{cases}
  x & \text{if } x \geq 0 \\
  0 & \text{if } x < 0
\end{cases}
= \max(0, x)
\]

% === Exercise 10: Real Document ===
% Problem: Short document with 3+ display equations, 5+ inline, Greek,
% fractions, roots, summation/integral, text in math.

\section*{Exercise 10: The Normal Distribution}

The normal (Gaussian) distribution is one of the most important probability
distributions in statistics and machine learning.

A random variable $X$ follows a normal distribution with mean $\mu$ and
standard deviation $\sigma$ if its probability density function is:
\begin{equation}
f(x) = \frac{1}{\sigma\sqrt{2\pi}} \exp\!\left( -\frac{(x - \mu)^2}{2\sigma^2} \right)
\end{equation}

The standard normal distribution has $\mu = 0$ and $\sigma = 1$, giving:
\begin{equation}
\varphi(z) = \frac{1}{\sqrt{2\pi}} \, e^{-z^2/2}
\end{equation}

The cumulative distribution function $\Phi(z)$ is defined as:
\begin{equation}
\Phi(z) = \int_{-\infty}^{z} \varphi(t) \, dt
  = \frac{1}{\sqrt{2\pi}} \int_{-\infty}^{z} e^{-t^2/2} \, dt
\end{equation}

The expected value $E[X] = \mu$ and the variance $\text{Var}(X) = \sigma^2$.
For a sample of $n$ observations, the sample mean $\bar{x}$ converges to
$\mu$ as $n \to \infty$ by the law of large numbers.

A key result is the Gaussian integral:
\begin{equation}
\int_{-\infty}^{\infty} e^{-\alpha x^2} \, dx = \sqrt{\frac{\pi}{\alpha}}
\quad \text{for } \alpha > 0
\end{equation}

\end{document}
