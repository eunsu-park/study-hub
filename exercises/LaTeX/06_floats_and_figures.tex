% Exercises for Lesson 06: Floats, Figures & Tables
% Topic: LaTeX
% Solutions to practice problems from the lesson.
% Compile: pdflatex exercises/LaTeX/06_floats_and_figures.tex

\documentclass[12pt, a4paper]{article}
\usepackage[utf8]{inputenc}
\usepackage[T1]{fontenc}
\usepackage{amsmath, amssymb}
\usepackage{graphicx}
\usepackage{subcaption}
\usepackage{float}
\usepackage{wrapfig}
\usepackage{booktabs}
\usepackage{hyperref}
\usepackage{lipsum}  % for dummy text

\title{Exercises for Lesson 06: Floats, Figures \& Tables}
\author{Study Project}
\date{}

\begin{document}
\maketitle

% Note: Since we do not have actual image files, we use \rule{}{} as
% placeholder rectangles to produce compilable, self-contained solutions.

% === Exercise 1: Basic Figure ===
% Problem: Three figures with different widths, captions, labels, and refs.

\section*{Exercise 1: Basic Figure}

Figure~\ref{fig:half} is half the text width.
Figure~\ref{fig:three-quarter} is 75\% wide.
Figure~\ref{fig:full} spans the full text width.

\begin{figure}[htbp]
  \centering
  \rule{0.5\textwidth}{3cm}
  \caption{A figure at 0.5\textbackslash textwidth}
  \label{fig:half}
\end{figure}

\begin{figure}[htbp]
  \centering
  \rule{0.75\textwidth}{3cm}
  \caption{A figure at 0.75\textbackslash textwidth}
  \label{fig:three-quarter}
\end{figure}

\begin{figure}[htbp]
  \centering
  \rule{\textwidth}{3cm}
  \caption{A figure at full text width}
  \label{fig:full}
\end{figure}

% === Exercise 2: Subfigures ===
% Problem: Four subfigures in a 2x2 grid with subcaptions and main caption.

\section*{Exercise 2: Subfigures}

Figure~\ref{fig:grid} shows four subfigures.
Specifically, Figure~\ref{fig:sub-a} is the top-left panel.

\begin{figure}[htbp]
  \centering
  \begin{subfigure}{0.45\textwidth}
    \centering
    \rule{\textwidth}{3cm}
    \caption{Panel A}
    \label{fig:sub-a}
  \end{subfigure}
  \hfill
  \begin{subfigure}{0.45\textwidth}
    \centering
    \rule{\textwidth}{3cm}
    \caption{Panel B}
    \label{fig:sub-b}
  \end{subfigure}

  \medskip

  \begin{subfigure}{0.45\textwidth}
    \centering
    \rule{\textwidth}{3cm}
    \caption{Panel C}
    \label{fig:sub-c}
  \end{subfigure}
  \hfill
  \begin{subfigure}{0.45\textwidth}
    \centering
    \rule{\textwidth}{3cm}
    \caption{Panel D}
    \label{fig:sub-d}
  \end{subfigure}
  \caption{Four subfigures arranged in a $2 \times 2$ grid}
  \label{fig:grid}
\end{figure}

% === Exercise 3: Mixed Floats ===
% Problem: 2 figures, 2 tables, cross-references, list of figures/tables.

\section*{Exercise 3: Mixed Floats}

See Figure~\ref{fig:mixed-a}, Figure~\ref{fig:mixed-b},
Table~\ref{tab:mixed-a}, and Table~\ref{tab:mixed-b}.

\begin{figure}[htbp]
  \centering
  \rule{0.6\textwidth}{3cm}
  \caption{First figure in mixed floats exercise}
  \label{fig:mixed-a}
\end{figure}

\begin{figure}[htbp]
  \centering
  \rule{0.6\textwidth}{3cm}
  \caption{Second figure in mixed floats exercise}
  \label{fig:mixed-b}
\end{figure}

\begin{table}[htbp]
  \centering
  \caption{First table in mixed floats exercise}
  \label{tab:mixed-a}
  \begin{tabular}{lcc}
    \toprule
    Item & Value 1 & Value 2 \\
    \midrule
    A & 10 & 20 \\
    B & 30 & 40 \\
    \bottomrule
  \end{tabular}
\end{table}

\begin{table}[htbp]
  \centering
  \caption{Second table in mixed floats exercise}
  \label{tab:mixed-b}
  \begin{tabular}{lcc}
    \toprule
    Method & Accuracy & Speed \\
    \midrule
    X & 95\% & 12 ms \\
    Y & 97\% & 18 ms \\
    \bottomrule
  \end{tabular}
\end{table}

\listoffigures
\listoftables

% === Exercise 4: Figure Sizing ===
% Problem: Same figure at different sizes.

\section*{Exercise 4: Figure Sizing}

\begin{figure}[htbp]
  \centering
  \rule{0.8\textwidth}{4cm}
  \caption{Width-based: 0.8\textbackslash textwidth}
\end{figure}

\begin{figure}[htbp]
  \centering
  \rule{8cm}{6cm}
  \caption{Height-based: 6cm tall (and 8cm wide as placeholder)}
\end{figure}

\begin{figure}[htbp]
  \centering
  \rule{6cm}{3.6cm}
  \caption{Scale-based: simulated 0.6 scale (60\% of 10cm $\times$ 6cm)}
\end{figure}

\begin{figure}[htbp]
  \centering
  \rule{10cm}{8cm}
  \caption{Fixed size: width=10cm, height=8cm, keepaspectratio}
\end{figure}

% === Exercise 5: Float Placement ===
% Problem: Experiment with placement specifiers h, t, b, p, H.

\section*{Exercise 5: Float Placement}

\begin{verbatim}
\begin{figure}[h]   % here
  \centering
  \includegraphics[width=0.5\textwidth]{image}
  \caption{Placed with [h]}
\end{figure}

\begin{figure}[t]   % top of page
  \centering
  \includegraphics[width=0.5\textwidth]{image}
  \caption{Placed with [t]}
\end{figure}

\begin{figure}[b]   % bottom of page
  \centering
  \includegraphics[width=0.5\textwidth]{image}
  \caption{Placed with [b]}
\end{figure}

\begin{figure}[p]   % float page
  \centering
  \includegraphics[width=0.5\textwidth]{image}
  \caption{Placed with [p]}
\end{figure}

\begin{figure}[H]   % exactly here (float package)
  \centering
  \includegraphics[width=0.5\textwidth]{image}
  \caption{Placed with [H]}
\end{figure}
\end{verbatim}

% === Exercise 6: Wrapped Figure ===
% Problem: wrapfig with text wrapping.

\section*{Exercise 6: Wrapped Figure}

\begin{wrapfigure}{r}{0.4\textwidth}
  \centering
  \rule{0.38\textwidth}{4cm}
  \caption{Wrapped figure}
\end{wrapfigure}

\lipsum[1]

% === Exercise 7: Side-by-Side Comparison ===
% Problem: subfigure vs minipage approaches.

\section*{Exercise 7: Side-by-Side Comparison}

\textbf{Approach A -- subfigure (shared number):}

\begin{figure}[htbp]
  \centering
  \begin{subfigure}{0.45\textwidth}
    \centering
    \rule{\textwidth}{3cm}
    \caption{Left subfigure}
  \end{subfigure}
  \hfill
  \begin{subfigure}{0.45\textwidth}
    \centering
    \rule{\textwidth}{3cm}
    \caption{Right subfigure}
  \end{subfigure}
  \caption{Two subfigures sharing one figure number}
\end{figure}

\textbf{Approach B -- minipage (separate numbers):}

\begin{figure}[htbp]
  \centering
  \begin{minipage}{0.45\textwidth}
    \centering
    \rule{\textwidth}{3cm}
    \caption{Left minipage figure}
  \end{minipage}
  \hfill
  \begin{minipage}{0.45\textwidth}
    \centering
    \rule{\textwidth}{3cm}
    \caption{Right minipage figure}
  \end{minipage}
\end{figure}

% === Exercise 8: Research Document ===
% Problem: Mini research paper with title, lists, 3 sections, 3 figures
% (1 with subfigures), 2 tables, cross-references.

\section*{Exercise 8: Mini Research Paper}

\begin{verbatim}
\documentclass{article}
\usepackage{graphicx, subcaption, booktabs, hyperref}

\title{A Mini Research Paper}
\author{Jane Doe}
\date{\today}

\begin{document}
\maketitle
\listoffigures
\listoftables

\section{Introduction}
We study three methods (Figure~\ref{fig:overview}).

\begin{figure}[htbp]
  \centering
  \begin{subfigure}{0.3\textwidth}
    \centering\rule{\textwidth}{2cm}
    \caption{Method A}\label{fig:a}
  \end{subfigure}\hfill
  \begin{subfigure}{0.3\textwidth}
    \centering\rule{\textwidth}{2cm}
    \caption{Method B}\label{fig:b}
  \end{subfigure}\hfill
  \begin{subfigure}{0.3\textwidth}
    \centering\rule{\textwidth}{2cm}
    \caption{Method C}\label{fig:c}
  \end{subfigure}
  \caption{Overview of methods}\label{fig:overview}
\end{figure}

\section{Results}
Table~\ref{tab:acc} and Table~\ref{tab:speed} show results.
Figure~\ref{fig:curve} shows the learning curve.

\begin{table}[htbp]
  \centering\caption{Accuracy}\label{tab:acc}
  \begin{tabular}{lc}\toprule
    Method & Accuracy \\\midrule
    A & 92\% \\ B & 95\% \\ C & 97\% \\\bottomrule
  \end{tabular}
\end{table}

\begin{table}[htbp]
  \centering\caption{Speed}\label{tab:speed}
  \begin{tabular}{lc}\toprule
    Method & Time (ms) \\\midrule
    A & 5 \\ B & 12 \\ C & 25 \\\bottomrule
  \end{tabular}
\end{table}

\begin{figure}[htbp]
  \centering\rule{0.6\textwidth}{4cm}
  \caption{Learning curve}\label{fig:curve}
\end{figure}

\begin{figure}[htbp]
  \centering\rule{0.6\textwidth}{4cm}
  \caption{Error analysis}\label{fig:error}
\end{figure}

\section{Conclusion}
Method C (Figure~\ref{fig:c}) achieves the best accuracy
(Table~\ref{tab:acc}) despite being slowest (Table~\ref{tab:speed}).

\end{document}
\end{verbatim}

\end{document}
