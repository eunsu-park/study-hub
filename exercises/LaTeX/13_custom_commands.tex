% Exercises for Lesson 13: Custom Commands & Environments
% Topic: LaTeX
% Solutions to practice problems from the lesson.
% Compile: pdflatex exercises/LaTeX/13_custom_commands.tex

\documentclass[12pt, a4paper]{article}
\usepackage[utf8]{inputenc}
\usepackage[T1]{fontenc}
\usepackage{amsmath, amssymb, amsthm}
\usepackage{xcolor}
\usepackage{xparse}
\usepackage{etoolbox}

\title{Exercises for Lesson 13: Custom Commands \& Environments}
\author{Study Project}
\date{}

% =====================================================================
% Exercise 1: Basic Math Shortcuts
% =====================================================================

% Set builder notation: \setbuilder{condition}{set}
\newcommand{\setbuilder}[2]{\left\{ #1 \;\middle|\; #2 \right\}}

% Probability, expectation, variance
\DeclareMathOperator{\Prob}{\mathbb{P}}
\newcommand{\ProbOf}[1]{\Prob\!\left( #1 \right)}
\newcommand{\Expect}[1]{\mathbb{E}\!\left[ #1 \right]}
\newcommand{\Var}[1]{\mathrm{Var}\!\left( #1 \right)}

% Big-O notation
\newcommand{\bigO}[1]{\mathcal{O}\!\left( #1 \right)}
\newcommand{\bigOmega}[1]{\Omega\!\left( #1 \right)}
\newcommand{\bigTheta}[1]{\Theta\!\left( #1 \right)}

% Number sets (reusable)
\newcommand{\R}{\mathbb{R}}
\newcommand{\N}{\mathbb{N}}
\newcommand{\Z}{\mathbb{Z}}
\newcommand{\C}{\mathbb{C}}

% =====================================================================
% Exercise 2: Custom warning environment
% =====================================================================

\newenvironment{warning}[1][Warning]{%
  \par\medskip
  \noindent
  \colorbox{red!10}{%
    \begin{minipage}{\dimexpr\textwidth-2\fboxsep}
      \textbf{\textcolor{red!70!black}{\textbullet\ #1}}
      \par\smallskip\small
}{%
    \end{minipage}%
  }%
  \par\medskip
}

% =====================================================================
% Exercise 3: Conditional compilation
% =====================================================================

\newtoggle{studentversion}
\toggletrue{studentversion}  % Change to \togglefalse for instructor

\newcommand{\version}[2]{%
  \iftoggle{studentversion}{#1}{#2}%
}

% =====================================================================
% Exercise 4: Derivative command (xparse)
% =====================================================================

\NewDocumentCommand{\deriv}{s O{1} m m}{%
  \IfBooleanTF{#1}{% starred: display style
    \dfrac{\mathrm{d}^{#2} #3}{\mathrm{d} #4^{#2}}%
  }{% unstarred: normal
    \frac{\mathrm{d}^{#2} #3}{\mathrm{d} #4^{#2}}%
  }%
}

% =====================================================================
% Exercise 6: Exercise environment with counter
% =====================================================================

\newcounter{exercisenum}[section]

\NewDocumentEnvironment{exercise}{O{medium}}{%
  \refstepcounter{exercisenum}%
  \par\medskip\noindent
  \textbf{Exercise~\theexercisenum}
  \ifstrequal{#1}{easy}{~(\textcolor{green!60!black}{Easy})}{}%
  \ifstrequal{#1}{medium}{~(\textcolor{orange}{Medium})}{}%
  \ifstrequal{#1}{hard}{~(\textcolor{red}{Hard})}{}%
  \textbf{.}\quad
}{%
  \par\medskip
}

% Theorem environment for Exercise 5
\newtheorem{theorem}{Theorem}

\begin{document}
\maketitle

% === Exercise 1 ===

\section*{Exercise 1: Basic Math Shortcuts}

Set builder notation:
\[
  \setbuilder{x \in \R}{x > 0}
\]

Probability, expectation, variance:
\[
  \ProbOf{X = 1} = 0.5, \qquad
  \Expect{X} = \mu, \qquad
  \Var{X} = \sigma^2
\]

Complexity notation:
\[
  \bigO{n^2}, \qquad
  \bigOmega{n}, \qquad
  \bigTheta{n \log n}
\]

\textbf{Definitions used:}
\begin{verbatim}
\newcommand{\setbuilder}[2]{%
  \left\{ #1 \;\middle|\; #2 \right\}}
\newcommand{\ProbOf}[1]{%
  \mathbb{P}\!\left( #1 \right)}
\newcommand{\Expect}[1]{%
  \mathbb{E}\!\left[ #1 \right]}
\newcommand{\Var}[1]{%
  \mathrm{Var}\!\left( #1 \right)}
\newcommand{\bigO}[1]{%
  \mathcal{O}\!\left( #1 \right)}
\end{verbatim}

% === Exercise 2 ===

\section*{Exercise 2: Custom Warning Environment}

\begin{warning}
  This is a default warning with no custom title.
  Be careful when using floating-point arithmetic.
\end{warning}

\begin{warning}[Deprecation Notice]
  The \texttt{eqnarray} environment is obsolete.
  Use \texttt{align} from \texttt{amsmath} instead.
\end{warning}

% === Exercise 3 ===

\section*{Exercise 3: Conditional Compilation}

\textbf{Problem:} Solve $\int_0^1 x^2 \, dx$.

\version{%
  % Student version: blank space for solution
  \vspace{2cm}
  \textit{(Write your solution here.)}
}{%
  % Instructor version: full solution
  \textbf{Solution:}
  \[
    \int_0^1 x^2 \, dx = \left[ \frac{x^3}{3} \right]_0^1
    = \frac{1}{3}
  \]
}

Currently compiled with
\texttt{studentversion=\iftoggle{studentversion}{true}{false}}.
Toggle the boolean in the preamble to switch.

% === Exercise 4 ===

\section*{Exercise 4: Derivative Command}

First derivative: $\deriv{f}{x}$

Second derivative: $\deriv[2]{f}{x}$

Display-style (starred): $\deriv*{f}{x}$

Display-style second derivative: $\deriv*[2]{f}{x}$

In display math:
\[
  \deriv[2]{u}{t} = c^2 \deriv[2]{u}{x}
\]

\textbf{Syntax:}
\begin{verbatim}
\NewDocumentCommand{\deriv}{s O{1} m m}{%
  \IfBooleanTF{#1}{%  starred -> display style
    \dfrac{\mathrm{d}^{#2} #3}{\mathrm{d} #4^{#2}}%
  }{%  unstarred -> normal
    \frac{\mathrm{d}^{#2} #3}{\mathrm{d} #4^{#2}}%
  }%
}
\end{verbatim}

% === Exercise 5 ===

\section*{Exercise 5: Personal Package}

A \texttt{mymath.sty} file would contain commands like those defined in
this document's preamble.  Below is the recommended file structure:

\begin{verbatim}
% mymath.sty - Personal math shortcuts
\NeedsTeXFormat{LaTeX2e}
\ProvidesPackage{mymath}[2024/01/01 Personal math]

\RequirePackage{amsmath, amssymb, amsthm}

% Number sets
\newcommand{\R}{\mathbb{R}}
\newcommand{\N}{\mathbb{N}}
\newcommand{\Z}{\mathbb{Z}}
\newcommand{\C}{\mathbb{C}}
\newcommand{\Q}{\mathbb{Q}}

% Operators
\DeclareMathOperator{\tr}{tr}
\DeclareMathOperator{\rank}{rank}
\DeclareMathOperator{\diag}{diag}
\DeclareMathOperator*{\argmin}{arg\,min}
\DeclareMathOperator*{\argmax}{arg\,max}

% Delimiters
\newcommand{\abs}[1]{\left| #1 \right|}
\newcommand{\norm}[1]{\left\| #1 \right\|}
\newcommand{\inner}[2]{\left\langle #1, #2 \right\rangle}

% Theorem environments
\newtheorem{mythm}{Theorem}[section]
\newtheorem{mylem}[mythm]{Lemma}

\endinput
\end{verbatim}

\textbf{Usage example:}

\begin{theorem}
  For any $v \in \R^n$, $\norm{v} \geq 0$ with equality if and only if
  $v = 0$.
\end{theorem}

\begin{proof}
  By definition, $\norm{v} = \sqrt{\sum_{i=1}^n v_i^2} \geq 0$.
  Equality holds iff every $v_i = 0$.
\end{proof}

% === Exercise 6 ===

\section{Exercise 6: Counter-Based Numbering}

\begin{exercise}[easy]
  Compute $2 + 3$.
\end{exercise}

\begin{exercise}[medium]
  Solve the quadratic equation $x^2 - 5x + 6 = 0$.
\end{exercise}

\begin{exercise}[hard]
  Prove that $\sqrt{2}$ is irrational.
\end{exercise}

\section{Another Section}

\begin{exercise}
  Counter resets: this is Exercise~\theexercisenum{} in the new section.
\end{exercise}

% === Exercise 7 ===

\section*{Exercise 7: Dynamic Content (Toggle)}

\begin{verbatim}
\newtoggle{matlab}
\toggletrue{matlab}  % or \togglefalse

\newcommand{\codestyle}[1]{%
  \iftoggle{matlab}{%
    {\color{blue}\ttfamily #1}%  MATLAB style
  }{%
    {\ttfamily #1}%              Generic style
  }%
}

% Usage:
\codestyle{x = linspace(0, 1, 100);}
\end{verbatim}

\newtoggle{matlab}
\toggletrue{matlab}

\newcommand{\codestyle}[1]{%
  \iftoggle{matlab}{%
    {\color{blue}\ttfamily #1}%
  }{%
    {\ttfamily #1}%
  }%
}

With \texttt{matlab=true}: \codestyle{x = linspace(0, 1, 100);}

\togglefalse{matlab}

With \texttt{matlab=false}: \codestyle{x = linspace(0, 1, 100);}

\end{document}
