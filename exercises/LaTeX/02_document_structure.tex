% Exercises for Lesson 02: Document Structure
% Topic: LaTeX
% Solutions to practice problems from the lesson.
% Compile: pdflatex exercises/LaTeX/02_document_structure.tex

\documentclass[12pt, a4paper]{article}
\usepackage[utf8]{inputenc}
\usepackage[T1]{fontenc}
\usepackage{amsmath, amssymb}
\usepackage{hyperref}

\title{Exercises for Lesson 02: Document Structure}
\author{Study Project}
\date{}

\begin{document}
\maketitle
\tableofcontents
\newpage

% === Exercise 1: Document Classes ===
% Problem: Create three documents -- article, report, book.

\section*{Exercise 1: Document Classes}

\subsection*{1a. Article}

\begin{verbatim}
\documentclass{article}
\title{Article Example}
\author{Author}
\date{\today}
\begin{document}
\maketitle
\section{Introduction}
Some text.
\subsection{Background}
Background information.
\section{Results}
Results here.
\end{document}
\end{verbatim}

\subsection*{1b. Report}

\begin{verbatim}
\documentclass{report}
\title{Report Example}
\author{Author}
\date{2024}
\begin{document}
\maketitle
\tableofcontents
\chapter{Introduction}
\section{Background}
Text.
\chapter{Methods}
\section{Approach}
Text.
\end{document}
\end{verbatim}

\subsection*{1c. Book}

\begin{verbatim}
\documentclass{book}
\title{Book Example}
\author{Author}
\date{2024}
\begin{document}
\frontmatter
\maketitle
\tableofcontents
\mainmatter
\chapter{Foundations}
\section{Basics}
Text.
\chapter{Advanced Topics}
\section{Deep Dive}
Text.
\backmatter
\chapter{Appendix}
Supplementary material.
\end{document}
\end{verbatim}

% === Exercise 2: Table of Contents ===
% Problem: Article with 3+ sections, 2+ subsections each, and a TOC.

\section{Exercise 2: Table of Contents}

This document itself demonstrates the solution.  Below is a minimal
standalone version:

\subsection{Quantum Mechanics}
\subsubsection{Wave-Particle Duality}
Light behaves as both a wave and a particle.

\subsection{Statistical Mechanics}
\subsubsection{Boltzmann Distribution}
The probability of a state is proportional to $e^{-E/kT}$.

\section{Mathematics}

\subsection{Algebra}
\subsubsection{Group Theory}
A group is a set with a binary operation satisfying closure, associativity,
identity, and invertibility.

\subsection{Analysis}
\subsubsection{Limits}
The limit of $f(x)$ as $x \to a$ is $L$ if for every $\epsilon > 0$ there
exists $\delta > 0$ such that $|f(x) - L| < \epsilon$ whenever
$0 < |x - a| < \delta$.

\section{Computer Science}

\subsection{Algorithms}
\subsubsection{Sorting}
Comparison-based sorting has a lower bound of $\Omega(n \log n)$.

\subsection{Data Structures}
\subsubsection{Trees}
Binary search trees allow $O(\log n)$ search on average.

% === Exercise 3: Options Exploration ===
% Problem: Same document with different option combinations.

\section*{Exercise 3: Options Exploration}

\begin{verbatim}
% Variant A
\documentclass[10pt, letterpaper, oneside]{article}
\begin{document}
\section{Test Section}
Sample text.
\end{document}

% Variant B
\documentclass[12pt, a4paper, twoside]{article}
\begin{document}
\section{Test Section}
Sample text.
\end{document}

% Variant C
\documentclass[11pt, a4paper, twocolumn]{article}
\begin{document}
\section{Test Section}
Sample text.
\end{document}
\end{verbatim}

% === Exercise 4: Unnumbered Sections ===
% Problem: Numbered + unnumbered sections, unnumbered in TOC.

\section*{Exercise 4: Unnumbered Sections}

\begin{verbatim}
\documentclass{article}
\begin{document}
\tableofcontents
\section{Introduction}
Content.
\section{Results}
Content.
\section*{Acknowledgments}
\addcontentsline{toc}{section}{Acknowledgments}
We thank the reviewers.
\section*{References}
\addcontentsline{toc}{section}{References}
[1] Knuth, D. The TeXbook, 1984.
\end{document}
\end{verbatim}

% === Exercise 5: Multi-File Document ===
% Problem: main.tex + intro.tex + methods.tex + conclusion.tex

\section*{Exercise 5: Multi-File Document}

\begin{verbatim}
% --- main.tex ---
\documentclass{article}
\title{Multi-File Example}
\author{Author}
\date{\today}
\begin{document}
\maketitle
\input{intro}
\input{methods}
\input{conclusion}
\end{document}

% --- intro.tex ---
\section{Introduction}
This document demonstrates multi-file organization.

% --- methods.tex ---
\section{Methods}
We use \LaTeX{} to typeset documents.

% --- conclusion.tex ---
\section{Conclusion}
Multi-file projects scale well for large documents.
\end{verbatim}

% === Exercise 6: Custom Commands ===
% Problem: Define \R, \dd, \vect commands.

\section*{Exercise 6: Custom Commands}

\newcommand{\R}{\mathbb{R}}
\newcommand{\dd}{\mathrm{d}}
\newcommand{\vect}[1]{\mathbf{#1}}

Below we demonstrate the custom commands:

The set of real numbers is $\R$.
The integral $\int f(x)\,\dd x$ uses an upright differential.
The vector $\vect{v} = (v_1, v_2, v_3) \in \R^3$ is bold.

\begin{verbatim}
% Preamble definitions:
\newcommand{\R}{\mathbb{R}}
\newcommand{\dd}{\mathrm{d}}
\newcommand{\vect}[1]{\mathbf{#1}}

% Usage:
$\R$, $\int f(x)\,\dd x$, $\vect{v}$
\end{verbatim}

% === Exercise 7: Complete Report ===
% Problem: Title page, abstract, TOC, 3 chapters, appendix, bibliography.

\section*{Exercise 7: Complete Report}

\begin{verbatim}
\documentclass[12pt, a4paper]{report}
\usepackage[utf8]{inputenc}
\usepackage{amsmath}
\usepackage{hyperref}

\title{A Technical Report on LaTeX}
\author{Jane Smith}
\date{2024}

\begin{document}
\maketitle
\begin{abstract}
This report demonstrates the structure of a complete LaTeX
report document with chapters, appendices, and a bibliography.
\end{abstract}
\tableofcontents

\chapter{Introduction}
LaTeX is a powerful typesetting system.

\chapter{Methodology}
We use the report class with 12pt font.

\chapter{Results}
The output is a well-structured PDF.

\appendix
\chapter{Supplementary Material}
Additional tables and derivations.

\begin{thebibliography}{9}
\bibitem{knuth} D. Knuth. \textit{The TeXbook}, 1984.
\bibitem{lamport} L. Lamport. \textit{LaTeX: A Document
  Preparation System}, 1994.
\end{thebibliography}
\end{document}
\end{verbatim}

\end{document}
