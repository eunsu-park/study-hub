% Exercises for Lesson 15: Automation and Build
% Topic: LaTeX
% Solutions to practice problems from the lesson.
% Compile: pdflatex exercises/LaTeX/15_automation_and_build.tex

\documentclass[12pt, a4paper]{article}
\usepackage[utf8]{inputenc}
\usepackage[T1]{fontenc}
\usepackage{amsmath, amssymb}
\usepackage{listings}
\usepackage{xcolor}
\usepackage{hyperref}

\lstset{
  basicstyle=\ttfamily\small,
  keywordstyle=\color{blue},
  commentstyle=\color{green!50!black},
  stringstyle=\color{red},
  frame=single,
  breaklines=true,
  showstringspaces=false
}

\title{Exercises for Lesson 15: Automation \& Build}
\author{Study Project}
\date{}

\begin{document}
\maketitle

% === Exercise 1: Compare Engines ===
% Problem: Compile with pdflatex, xelatex, lualatex.

\section*{Exercise 1: Compare Engines}

Create a test document (\texttt{engine\_test.tex}):

\begin{lstlisting}[language=TeX]
\documentclass{article}
\usepackage{fontspec}  % requires xelatex or lualatex
\setmainfont{Times New Roman}  % system font

\begin{document}
\section{Unicode Test}
Greek: Alpha Beta Gamma Delta

Chinese: Test text here

Emoji: (note: emoji require special font setup)

\section{Math}
\[ E = mc^2 \]
\end{document}
\end{lstlisting}

\textbf{Compilation commands:}
\begin{lstlisting}[language=bash]
# pdflatex -- FAILS on fontspec
pdflatex engine_test.tex

# xelatex -- works with fontspec + system fonts
xelatex engine_test.tex

# lualatex -- works with fontspec + system fonts
lualatex engine_test.tex
\end{lstlisting}

\textbf{Results:}
\begin{itemize}
  \item \texttt{pdflatex}: Cannot use \texttt{fontspec}; remove it or use
    \texttt{mathptmx} instead for Times-like font
  \item \texttt{xelatex}: Works.  Typically faster than lualatex
  \item \texttt{lualatex}: Works.  Slightly slower but supports Lua scripting
\end{itemize}

% === Exercise 2: Set Up latexmk ===
% Problem: .latexmkrc with pdflatex, output dir, PDF viewer.

\section*{Exercise 2: latexmk Configuration}

Create \texttt{\textasciitilde/.latexmkrc}:

\begin{lstlisting}[language=Perl]
# Default to pdflatex
$pdf_mode = 1;
$pdflatex = 'pdflatex -interaction=nonstopmode -synctex=1 %O %S';

# Output directory
$out_dir = 'build';
$aux_dir = 'build';

# PDF viewer (macOS)
$pdf_previewer = 'open -a Preview %S';

# For Linux:
# $pdf_previewer = 'evince %S';

# For Windows:
# $pdf_previewer = 'start %S';

# Clean extensions
@generated_exts = qw(aux log toc lof lot out bbl blg synctex.gz fdb_latexmk fls);
\end{lstlisting}

\textbf{Usage:}
\begin{lstlisting}[language=bash]
# Build
latexmk -pdf document.tex

# Watch for changes (auto-rebuild)
latexmk -pdf -pvc document.tex

# Clean auxiliary files
latexmk -c

# Clean everything including PDF
latexmk -C
\end{lstlisting}

% === Exercise 3: Create Makefile ===
% Problem: Makefile for report.tex with chapters and bibliography.

\section*{Exercise 3: Makefile}

\begin{lstlisting}[language=make]
# Makefile for LaTeX project
MAIN    = report
LATEXMK = latexmk
FLAGS   = -pdf -interaction=nonstopmode

# Source files
SOURCES = $(MAIN).tex chapters/ch1.tex chapters/ch2.tex \
          chapters/ch3.tex references.bib

.PHONY: all clean watch view

all: $(MAIN).pdf

$(MAIN).pdf: $(SOURCES)
	$(LATEXMK) $(FLAGS) $(MAIN).tex

clean:
	$(LATEXMK) -c $(MAIN).tex
	rm -f $(MAIN).bbl $(MAIN).run.xml

watch:
	$(LATEXMK) $(FLAGS) -pvc $(MAIN).tex

view: $(MAIN).pdf
	open $(MAIN).pdf  # macOS; use xdg-open on Linux
\end{lstlisting}

% === Exercise 4: Git Workflow ===
% Problem: Git init, .gitignore, 3 commits, latexdiff.

\section*{Exercise 4: Git Workflow}

\textbf{Step 1: Initialize and create .gitignore:}

\begin{lstlisting}[language=bash]
mkdir latex-project && cd latex-project
git init

cat > .gitignore << 'EOF'
# LaTeX auxiliary files
*.aux
*.log
*.toc
*.lof
*.lot
*.out
*.bbl
*.blg
*.bcf
*.run.xml
*.synctex.gz
*.fdb_latexmk
*.fls

# Output
*.pdf

# Build directory
build/

# Editor files
*.swp
*~
.vscode/
EOF

git add .gitignore
git commit -m "Initial commit with .gitignore"
\end{lstlisting}

\textbf{Step 2: Three commits:}

\begin{lstlisting}[language=bash]
# Create initial document
cat > paper.tex << 'EOF'
\documentclass{article}
\title{My Paper}
\author{Author}
\begin{document}
\maketitle
\section{Introduction}
This is the first version.
\end{document}
EOF
git add paper.tex
git commit -m "Add initial paper with introduction"

# Add methods section
# (edit paper.tex to add \section{Methods} ...)
git add paper.tex
git commit -m "Add methods section"

# Add results section
# (edit paper.tex to add \section{Results} ...)
git add paper.tex
git commit -m "Add results section"
\end{lstlisting}

\textbf{Step 3: Use latexdiff:}

\begin{lstlisting}[language=bash]
# Compare first and last commits
git show HEAD~2:paper.tex > old.tex
git show HEAD:paper.tex > new.tex

latexdiff old.tex new.tex > diff.tex
pdflatex diff.tex

# Cleanup
rm old.tex new.tex
\end{lstlisting}

% === Exercise 5: GitHub Actions ===
% Problem: CI to build and upload PDF.

\section*{Exercise 5: GitHub Actions}

Create \texttt{.github/workflows/build.yml}:

\begin{lstlisting}[language=yaml]
name: Build LaTeX
on:
  push:
    branches: [ main ]
  pull_request:
    branches: [ main ]

jobs:
  build:
    runs-on: ubuntu-latest
    steps:
      - name: Checkout
        uses: actions/checkout@v4

      - name: Compile LaTeX
        uses: xu-cheng/latex-action@v3
        with:
          root_file: paper.tex
          latexmk_use_xelatex: false

      - name: Upload PDF
        uses: actions/upload-artifact@v4
        with:
          name: paper-pdf
          path: paper.pdf
\end{lstlisting}

% === Exercise 6: Docker Build ===
% Problem: Dockerfile with TeX Live, compare output.

\section*{Exercise 6: Docker Build}

\begin{lstlisting}[language=docker]
FROM texlive/texlive:latest

WORKDIR /doc
COPY . .

CMD ["latexmk", "-pdf", "paper.tex"]
\end{lstlisting}

\textbf{Build and run:}
\begin{lstlisting}[language=bash]
# Build image
docker build -t latex-builder .

# Run container and copy PDF out
docker run --rm -v $(pwd)/output:/output latex-builder \
  sh -c "latexmk -pdf paper.tex && cp paper.pdf /output/"

# Compare with local build
diff <(md5sum paper.pdf) <(md5sum output/paper.pdf)
# Note: PDFs may differ due to timestamps in metadata
\end{lstlisting}

% === Exercise 7: Full Workflow ===
% Problem: Complete project with latexmk + Makefile + Git + CI + spell check.

\section*{Exercise 7: Full Workflow}

\textbf{Project structure:}
\begin{lstlisting}
project/
  paper.tex
  chapters/
    ch1.tex
    ch2.tex
  references.bib
  .latexmkrc
  Makefile
  .gitignore
  .github/
    workflows/
      build.yml
  .vscode/
    settings.json
\end{lstlisting}

\textbf{VS Code spell checking (\texttt{.vscode/settings.json}):}
\begin{lstlisting}[language=json]
{
  "cSpell.language": "en",
  "cSpell.enabledLanguageIds": ["latex"],
  "cSpell.words": [
    "amsmath", "amssymb", "booktabs",
    "hyperref", "cleveref", "biblatex"
  ],
  "latex-workshop.latex.recipe.default": "latexmk (pdf)"
}
\end{lstlisting}

\textbf{End-to-end test:}
\begin{lstlisting}[language=bash]
# 1. Clone and setup
git clone <repo-url> && cd project

# 2. Local build
make all          # builds paper.pdf

# 3. Watch mode (live rebuild)
make watch        # rebuilds on file save

# 4. Spell check
aspell -t -c paper.tex

# 5. Lint
chktex paper.tex

# 6. Push to trigger CI
git push origin main

# 7. Verify CI pass on GitHub
gh run list --limit 1
\end{lstlisting}

\end{document}
