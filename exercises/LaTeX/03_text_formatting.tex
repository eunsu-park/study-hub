% Exercises for Lesson 03: Text Formatting
% Topic: LaTeX
% Solutions to practice problems from the lesson.
% Compile: pdflatex exercises/LaTeX/03_text_formatting.tex

\documentclass[12pt, a4paper]{article}
\usepackage[utf8]{inputenc}
\usepackage[T1]{fontenc}
\usepackage{amsmath, amssymb}
\usepackage{xcolor}
\usepackage{enumitem}
\usepackage{listings}
\usepackage{hyperref}

\title{Exercises for Lesson 03: Text Formatting}
\author{Study Project}
\date{}

\begin{document}
\maketitle

% === Exercise 1: Font Styles ===
% Problem: Demonstrate bold, italic, monospace, combinations, small caps,
% and at least 5 font sizes.

\section*{Exercise 1: Font Styles}

\textbf{Bold text} demonstrates emphasis for important terms.

\textit{Italic text} is used for titles and foreign words.

\texttt{Monospace text} is used for code and filenames.

\textbf{\textit{Bold italic}} combines two styles.

\texttt{\textbf{Bold monospace}} is useful for code emphasis.

\underline{\textbf{Bold underlined}} draws strong attention.

\textsc{Small Capitals} are often used for acronyms like \textsc{nasa}.

\medskip

Five different font sizes:

{\tiny This is tiny text.}

{\footnotesize This is footnote-size text.}

{\normalsize This is normal-size text.}

{\Large This is Large text.}

{\Huge This is Huge text.}

% === Exercise 2: Colors ===
% Problem: 3 predefined colors, 3 custom RGB, colored background,
% colored heading.

\section*{Exercise 2: Colors}

\definecolor{steelblue}{RGB}{70, 130, 180}
\definecolor{coral}{RGB}{255, 127, 80}
\definecolor{forestgreen}{RGB}{34, 139, 34}

Predefined colors: \textcolor{red}{Red text}, \textcolor{blue}{Blue text},
\textcolor{magenta}{Magenta text}.

Custom colors: \textcolor{steelblue}{Steel Blue text},
\textcolor{coral}{Coral text}, \textcolor{forestgreen}{Forest Green text}.

Text with \colorbox{yellow}{yellow background}.

\textcolor{steelblue}{\Large\bfseries A Colored Section Heading}

This paragraph follows the colored heading.

% === Exercise 3: Lists ===
% Problem: Bulleted, numbered, description, nested (3 levels), custom labels.

\section*{Exercise 3: Lists}

\textbf{Bulleted list:}
\begin{itemize}
    \item First item
    \item Second item
    \item Third item
\end{itemize}

\textbf{Numbered list:}
\begin{enumerate}
    \item First step
    \item Second step
    \item Third step
\end{enumerate}

\textbf{Description list:}
\begin{description}
    \item[LaTeX] A document preparation system
    \item[TeX] The underlying typesetting engine
    \item[PDF] Portable Document Format
\end{description}

\textbf{Nested list (3 levels):}
\begin{enumerate}
    \item First level
    \begin{itemize}
        \item Second level -- bullet A
        \item Second level -- bullet B
        \begin{enumerate}
            \item Third level -- numbered
            \item Third level -- numbered
        \end{enumerate}
    \end{itemize}
    \item First level again
\end{enumerate}

\textbf{Custom labels:}
\begin{itemize}[label=$\star$]
    \item Star bullet
    \item Another star bullet
\end{itemize}

\begin{enumerate}[label=\Roman*.]
    \item Roman numeral I
    \item Roman numeral II
    \item Roman numeral III
\end{enumerate}

% === Exercise 4: Quotations ===
% Problem: quote, quotation, verse, nested inline quotes.

\section*{Exercise 4: Quotations}

\textbf{Short quote:}
\begin{quote}
The only way to do great work is to love what you do.
\end{quote}

\textbf{Long quotation:}
\begin{quotation}
This is the first paragraph of a longer quotation.  It demonstrates that
the quotation environment indents the first line of each paragraph.

This is the second paragraph.  Notice the indentation at the beginning of
this paragraph, which distinguishes it from the \texttt{quote} environment.
\end{quotation}

\textbf{Verse:}
\begin{verse}
Roses are red, \\
Violets are blue, \\
\LaTeX{} is great, \\
And so are you.
\end{verse}

\textbf{Nested inline quotes:}
``She said, `Hello!' to me.''

% === Exercise 5: Verbatim and Code ===
% Problem: inline verbatim, multi-line verbatim, listings for Python,
% special characters verbatim.

\section*{Exercise 5: Verbatim and Code}

\textbf{Inline verbatim:} The command \verb|\LaTeX| produces the \LaTeX{} logo.

\textbf{Multi-line verbatim:}
\begin{verbatim}
def hello(name):
    print(f"Hello, {name}!")

hello("World")
\end{verbatim}

\textbf{Code listing (Python):}

\lstset{
    language=Python,
    basicstyle=\ttfamily\small,
    keywordstyle=\color{blue},
    commentstyle=\color{green!50!black},
    stringstyle=\color{red},
    numbers=left,
    numberstyle=\tiny\color{gray},
    frame=single,
    breaklines=true
}

\begin{lstlisting}
def fibonacci(n):
    """Return the nth Fibonacci number."""
    if n <= 1:
        return n
    return fibonacci(n - 1) + fibonacci(n - 2)

for i in range(10):
    print(fibonacci(i))
\end{lstlisting}

\textbf{Special characters in verbatim:}
\begin{verbatim}
Reserved characters: # $ % & _ { } \ ^ ~
\end{verbatim}

% === Exercise 6: Special Characters ===
% Problem: All reserved characters, three dash types, accents, symbols.

\section*{Exercise 6: Special Characters}

\textbf{Reserved characters:}
\textbackslash{} \{ \} \$ \& \% \# \_ \^{} \~{}

\textbf{Dash types:}
\begin{description}
    \item[Hyphen (-)] daughter-in-law
    \item[En-dash (--)] pages 10--20
    \item[Em-dash (---)] An interruption---like this---in a sentence
\end{description}

\textbf{Accented characters:}
Caf\'{e}, na\"{\i}ve, Z\"{u}rich, S\~{a}o Paulo

\textbf{Symbols:}
\copyright{} 2024, \textregistered{}, \texttrademark{}, \S, \P, \dag, \ddag, \pounds

% === Exercise 7: Spacing and Alignment ===
% Problem: Custom horizontal/vertical spacing, centered, left-aligned,
% right-aligned, \hfill title page.

\section*{Exercise 7: Spacing and Alignment}

\textbf{Custom horizontal spacing:}
Word1\hspace{2cm}Word2\hspace{1cm}Word3

\textbf{Custom vertical spacing:}

Text before vertical skip.

\vspace{1.5cm}

Text after 1.5cm vertical skip.

\textbf{Centered paragraph:}
\begin{center}
This paragraph is centered.  Every line is centered regardless of length.
\end{center}

\textbf{Left-aligned (no justification):}
\begin{flushleft}
This paragraph is left-aligned without right justification.
The right margin is ragged.
\end{flushleft}

\textbf{Right-aligned:}
\begin{flushright}
This paragraph is right-aligned.
The left margin is ragged.
\end{flushright}

\textbf{Title page with \texttt{\textbackslash hfill}:}

\begin{center}
{\LARGE\bfseries My Document Title}
\end{center}
\hfill {\itshape Jane Doe}

\hfill \today

% === Exercise 8: Footnotes ===
% Problem: 3 auto-numbered footnotes, custom number, footnotemark/text.

\section*{Exercise 8: Footnotes}

This sentence has a footnote.\footnote{This is the first footnote.}
Here is another.\footnote{Second footnote with automatic numbering.}
And a third.\footnote{Third footnote -- still automatic.}

This footnote has a custom number.\footnote[42]{This is footnote number 42.}

Here is a footnote mark using the two-step approach.\footnotemark

\footnotetext{This footnote text is placed separately using
\texttt{\textbackslash footnotetext}.}

% === Exercise 9: Complete Styled Document ===
% Problem: Combine title, sections, lists, colors, code, footnotes, quotation.

\section*{Exercise 9: Complete Styled Document}

{\Large\textcolor{steelblue}{\bfseries A Mini Styled Document}}

\medskip

\textsf{This section uses the sans-serif font family.}

\begin{itemize}
    \item Bullet point one
    \item Bullet point two
\end{itemize}

\begin{enumerate}
    \item Numbered item one
    \item Numbered item two
\end{enumerate}

\begin{description}
    \item[Term] Definition of the term
\end{description}

Here is some \textcolor{coral}{coral-colored text} and a
\colorbox{yellow!30}{lightly highlighted phrase}.

\begin{verbatim}
print("Hello from verbatim!")
\end{verbatim}

\begin{quote}
The best way to predict the future is to create it.\footnote{Attributed to
Peter Drucker.}
\end{quote}

And a second footnote for good measure.\footnote{This wraps up the exercise.}

% === Exercise 10: Real-World Application (Resume) ===
% Problem: Resume using bold headings, italic dates, bullet lists,
% custom spacing, footnote for contact info.

\section*{Exercise 10: Resume}

\begin{center}
{\LARGE\bfseries Jane Doe}\footnote{Email: jane.doe@example.com,
Phone: +1-555-0123}
\end{center}

\vspace{0.3cm}

{\large\bfseries Education}
\vspace{0.1cm}
\hrule
\vspace{0.3cm}
\textbf{M.Sc. Computer Science} \hfill \textit{2022--2024} \\
University of Technology

\textbf{B.Sc. Mathematics} \hfill \textit{2018--2022} \\
State University

\vspace{0.4cm}

{\large\bfseries Work Experience}
\vspace{0.1cm}
\hrule
\vspace{0.3cm}
\textbf{Software Engineer} \hfill \textit{Jun 2024 -- Present} \\
Tech Company, Inc.
\begin{itemize}[noitemsep]
    \item Developed backend services using Python and PostgreSQL
    \item Improved API response times by 40\%
    \item Led migration to containerized deployments
\end{itemize}

\textbf{Research Assistant} \hfill \textit{Sep 2022 -- May 2024} \\
University of Technology
\begin{itemize}[noitemsep]
    \item Published two papers on machine learning optimization
    \item Maintained the lab's GPU cluster infrastructure
\end{itemize}

\end{document}
