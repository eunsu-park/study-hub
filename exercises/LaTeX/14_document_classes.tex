% Exercises for Lesson 14: Document Classes & Templates
% Topic: LaTeX
% Solutions to practice problems from the lesson.
% Compile: pdflatex exercises/LaTeX/14_document_classes.tex

\documentclass[12pt, a4paper]{article}
\usepackage[utf8]{inputenc}
\usepackage[T1]{fontenc}
\usepackage{amsmath, amssymb}
\usepackage{hyperref}

\title{Exercises for Lesson 14: Document Classes \& Templates}
\author{Study Project}
\date{}

\begin{document}
\maketitle

% === Exercise 1: Compare Classes ===
% Problem: Same document in article, scrartcl, memoir.

\section*{Exercise 1: Compare Classes}

\textbf{Version A -- article:}
\begin{verbatim}
\documentclass{article}
\usepackage{amsmath}
\title{Comparison Document}
\author{Jane Doe}
\date{\today}
\begin{document}
\maketitle
\section{Introduction}
This is an introduction.
\section{Theory}
The equation $E = mc^2$ is famous:
\[ E = mc^2 \]
\end{document}
\end{verbatim}

\textbf{Version B -- scrartcl (KOMA-Script):}
\begin{verbatim}
\documentclass{scrartcl}
\usepackage{amsmath}
\title{Comparison Document}
\author{Jane Doe}
\date{\today}
\begin{document}
\maketitle
\section{Introduction}
This is an introduction.
\section{Theory}
The equation $E = mc^2$ is famous:
\[ E = mc^2 \]
\end{document}
\end{verbatim}

\textbf{Version C -- memoir:}
\begin{verbatim}
\documentclass{memoir}
\usepackage{amsmath}
\title{Comparison Document}
\author{Jane Doe}
\date{\today}
\begin{document}
\maketitle
\chapter{Introduction}
This is an introduction.
\chapter{Theory}
The equation $E = mc^2$ is famous:
\[ E = mc^2 \]
\end{document}
\end{verbatim}

\textbf{Key differences:}
\begin{itemize}
  \item \texttt{scrartcl}: Larger default section headings, European typography
    defaults, auto-calculated type area
  \item \texttt{memoir}: Uses \verb|\chapter| as top-level division,
    more built-in customization hooks
  \item \texttt{article}: Standard defaults, most widely compatible
\end{itemize}

% === Exercise 2: KOMA Options ===
% Problem: scrartcl with 11pt, A4, parskip=half, headings=big,
% auto type area, 3+ sections.

\section*{Exercise 2: KOMA-Script Options}

\begin{verbatim}
\documentclass[
  11pt,
  a4paper,
  parskip=half,
  headings=big
]{scrartcl}
\usepackage{typearea}
\KOMAoptions{DIV=calc}  % auto-calculate type area

\title{KOMA-Script Document}
\author{Author Name}
\date{\today}

\begin{document}
\maketitle

\section{First Section}
This document uses KOMA-Script options for modern European
typography. Notice the lack of paragraph indentation ---
parskip=half adds half a line of space between paragraphs.

This is a second paragraph demonstrating the spacing.

\section{Second Section}
The headings=big option makes section titles larger and more
prominent than the article class defaults.

\section{Third Section}
The type area is auto-calculated with DIV=calc, which
optimizes the text width for the chosen font size and paper.

\end{document}
\end{verbatim}

% === Exercise 3: Create a CV ===
% Problem: moderncv with personal info, education, experience, skills.

\section*{Exercise 3: CV with moderncv}

\begin{verbatim}
\documentclass[11pt,a4paper]{moderncv}
\moderncvstyle{classic}
\moderncvcolor{blue}
\usepackage[margin=2cm]{geometry}

\name{Jane}{Doe}
\email{jane.doe@example.com}
\phone[mobile]{+1-555-0123}
\social[linkedin]{janedoe}

\begin{document}
\makecvtitle

\section{Education}
\cventry{2022--2024}{M.Sc. Computer Science}
  {MIT}{Cambridge, MA}{\textit{GPA: 3.9/4.0}}
  {Thesis: Deep Learning for Climate Modeling}
\cventry{2018--2022}{B.Sc. Mathematics}
  {Stanford}{Stanford, CA}{\textit{GPA: 3.8/4.0}}
  {Minor in Physics}

\section{Experience}
\cventry{Jun 2024--Present}{Machine Learning Engineer}
  {DeepMind}{London, UK}{}
  {Developing large language models for scientific research.}
\cventry{May--Aug 2023}{Research Intern}
  {Google Brain}{Mountain View, CA}{}
  {Implemented efficient attention mechanisms for long-context
   models.}

\section{Skills}
\cvitem{Languages}{Python, C++, Julia, Rust}
\cvitem{Frameworks}{PyTorch, JAX, TensorFlow}
\cvitem{Tools}{Git, Docker, Kubernetes, LaTeX}

\end{document}
\end{verbatim}

% === Exercise 4: Thesis Title Page ===
% Problem: Custom title page with logo placeholder, title, author,
% degree, department, date.

\section*{Exercise 4: Thesis Title Page}

\begin{verbatim}
\documentclass[12pt]{report}
\usepackage{graphicx}
\begin{document}
\begin{titlepage}
  \centering
  \vspace*{1cm}

  % University logo placeholder
  \rule{3cm}{3cm}

  \vspace{1.5cm}

  {\LARGE\bfseries
    Deep Learning Approaches for\\
    Climate Prediction
  }

  \vspace{1cm}

  {\large\itshape Jane Doe}

  \vfill

  A thesis submitted for the degree of\\
  \textbf{Master of Science}

  \vspace{1cm}

  Department of Computer Science\\
  Massachusetts Institute of Technology

  \vspace{1cm}

  \today

\end{titlepage}

\tableofcontents

\chapter{Introduction}
Content begins here ...

\end{document}
\end{verbatim}

% === Exercise 5: Conference Template ===
% Problem: IEEE/ACM template, 2-page mock paper.

\section*{Exercise 5: Conference Template}

\begin{verbatim}
\documentclass[conference]{IEEEtran}
\usepackage{amsmath, graphicx, booktabs}
\usepackage[backend=biber,style=ieee]{biblatex}
\addbibresource{references.bib}

\title{A Novel Approach to Image Classification}
\author{
  \IEEEauthorblockN{Jane Doe}
  \IEEEauthorblockA{Department of CS, MIT\\
    Cambridge, MA 02139\\
    jane.doe@mit.edu}
}

\begin{document}
\maketitle

\begin{abstract}
We present a novel deep learning approach for image
classification that achieves state-of-the-art results
on benchmark datasets.
\end{abstract}

\section{Introduction}
Image classification is a fundamental task in computer
vision \parencite{lecun1998}.

\section{Methods}
Our model uses a modified ResNet architecture.

\begin{figure}[htbp]
  \centering
  \rule{0.8\columnwidth}{3cm}
  \caption{Model architecture overview.}
\end{figure}

\section{Results}
Table~I shows results on CIFAR-10.

\begin{table}[htbp]
  \centering
  \caption{CIFAR-10 Results}
  \begin{tabular}{lc}
    \toprule
    Model & Accuracy \\
    \midrule
    Baseline & 92.3\% \\
    Ours & 96.1\% \\
    \bottomrule
  \end{tabular}
\end{table}

\printbibliography

\end{document}
\end{verbatim}

% === Exercise 6: Letter ===
% Problem: Formal letter using scrlttr2.

\section*{Exercise 6: Formal Letter with scrlttr2}

\begin{verbatim}
\documentclass[
  fromalign=right,
  fromphone,
  fromemail,
  parskip=half
]{scrlttr2}

\setkomavar{fromname}{Jane Doe}
\setkomavar{fromaddress}{123 Main Street\\
  Cambridge, MA 02139}
\setkomavar{fromphone}{+1-555-0123}
\setkomavar{fromemail}{jane.doe@example.com}
\setkomavar{subject}{Application for Research Position}

\begin{document}

\begin{letter}{%
  Prof.\ John Smith\\
  Department of Computer Science\\
  Stanford University\\
  Stanford, CA 94305}

\opening{Dear Professor Smith,}

I am writing to express my interest in the research
assistant position in your computational physics group.
I recently completed my Master's degree at MIT, where
my thesis focused on deep learning for climate modeling.

During my studies I developed strong skills in both
theoretical machine learning and scientific computing.
I believe my background would be a valuable addition
to your team's work on physics-informed neural networks.

\closing{Sincerely,}

\end{letter}

\end{document}
\end{verbatim}

\end{document}
