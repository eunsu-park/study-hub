% Exercises for Lesson 08: Cross-References & Citations
% Topic: LaTeX
% Solutions to practice problems from the lesson.
% Compile: pdflatex exercises/LaTeX/08_cross_references.tex
%          (run twice for cross-references to resolve)

\documentclass[12pt, a4paper]{article}
\usepackage[utf8]{inputenc}
\usepackage[T1]{fontenc}
\usepackage{amsmath, amssymb}
\usepackage{graphicx}
\usepackage{booktabs}
\usepackage[colorlinks=true,linkcolor=blue,citecolor=green,urlcolor=red]{hyperref}

\title{Exercises for Lesson 08: Cross-References \& Citations}
\author{Study Project}
\date{}

\begin{document}
\maketitle

% === Exercise 1: Cross-References ===
% Problem: 3 sections, 2 figures, 2 tables, 3 equations, with \autoref.

\section{Introduction}
\label{sec:intro}

This document demonstrates cross-referencing.
\autoref{sec:methods} describes the methodology.
\autoref{sec:results} presents the results.

\section{Methods}
\label{sec:methods}

The governing equation is:
\begin{equation}
  E = mc^2
  \label{eq:einstein}
\end{equation}

The wave equation is:
\begin{equation}
  \frac{\partial^2 u}{\partial t^2} = c^2 \nabla^2 u
  \label{eq:wave}
\end{equation}

The heat equation is:
\begin{equation}
  \frac{\partial u}{\partial t} = \alpha \nabla^2 u
  \label{eq:heat}
\end{equation}

See \autoref{eq:einstein}, \autoref{eq:wave}, and \autoref{eq:heat}.

\autoref{fig:method} shows the experimental setup.
\autoref{tab:params} lists the parameters.

\begin{figure}[htbp]
  \centering
  \rule{0.6\textwidth}{3cm}
  \caption{Experimental setup}
  \label{fig:method}
\end{figure}

\begin{table}[htbp]
  \centering
  \caption{Model parameters}
  \label{tab:params}
  \begin{tabular}{lc}
    \toprule
    Parameter & Value \\
    \midrule
    Learning rate & 0.001 \\
    Batch size & 32 \\
    \bottomrule
  \end{tabular}
\end{table}

\section{Results}
\label{sec:results}

\autoref{fig:results} presents the accuracy curve.
\autoref{tab:results} summarizes the final metrics.

\begin{figure}[htbp]
  \centering
  \rule{0.6\textwidth}{3cm}
  \caption{Accuracy over training epochs}
  \label{fig:results}
\end{figure}

\begin{table}[htbp]
  \centering
  \caption{Final performance metrics}
  \label{tab:results}
  \begin{tabular}{lc}
    \toprule
    Metric & Value \\
    \midrule
    Accuracy & 95.7\% \\
    F1 Score & 0.953 \\
    \bottomrule
  \end{tabular}
\end{table}

As discussed in \autoref{sec:intro}, the energy--mass relation
(\autoref{eq:einstein}) is fundamental.

% === Exercise 2: Bibliography ===
% Problem: .bib file with 5 entries, cite all.
% Since this must be self-contained, we use thebibliography.

\section*{Exercise 2: Bibliography}

The solution requires a \texttt{.bib} file and \texttt{natbib}.
Below we show the structure using \texttt{thebibliography} so the
file compiles standalone.

Einstein's theory~\cite{einstein} is foundational.
The TeXbook~\cite{knuth} introduced modern typesetting.
CNNs~\cite{lecun} and Transformers~\cite{vaswani} changed deep learning.
Reinforcement learning~\cite{sutton} provides a framework for
sequential decision-making.

\begin{verbatim}
% references.bib (use this with natbib/biblatex):
@article{einstein1905, author={Einstein, A.},
  title={On the Electrodynamics of Moving Bodies},
  journal={Annalen der Physik}, year={1905}}
@book{knuth1984, author={Knuth, D.},
  title={The TeXbook}, publisher={Addison-Wesley}, year={1984}}
@inproceedings{lecun1998, author={LeCun, Y. and others},
  title={Gradient-based learning ...},
  booktitle={Proc. IEEE}, year={1998}}
@misc{vaswani2017, author={Vaswani, A. and others},
  title={Attention is All You Need}, year={2017}}
@book{sutton2018, author={Sutton, R. and Barto, A.},
  title={Reinforcement Learning}, publisher={MIT Press},
  year={2018}}
\end{verbatim}

\begin{thebibliography}{9}
\bibitem{einstein} A. Einstein. ``On the Electrodynamics of Moving Bodies.'' \textit{Annalen der Physik}, 1905.
\bibitem{knuth} D. Knuth. \textit{The TeXbook}. Addison-Wesley, 1984.
\bibitem{lecun} Y. LeCun et al. ``Gradient-based learning applied to document recognition.'' \textit{Proc. IEEE}, 1998.
\bibitem{vaswani} A. Vaswani et al. ``Attention is All You Need.'' arXiv:1706.03762, 2017.
\bibitem{sutton} R. Sutton and A. Barto. \textit{Reinforcement Learning: An Introduction}. MIT Press, 2018.
\end{thebibliography}

% === Exercise 3: BibLaTeX ===
% Problem: Convert to BibLaTeX with IEEE style (shown as verbatim).

\section*{Exercise 3: BibLaTeX Conversion}

\begin{verbatim}
\documentclass{article}
\usepackage[backend=biber,style=ieee]{biblatex}
\addbibresource{references.bib}

\begin{document}
Einstein \parencite{einstein1905} ...
Knuth \parencite{knuth1984} ...
\printbibliography
\end{document}

% Compile:
% pdflatex document.tex
% biber document
% pdflatex document.tex
\end{verbatim}

% === Exercise 4: Multiple Reference Types ===
% Problem: Section by number/name, page ref, equation range, multiple figs.

\section*{Exercise 4: Multiple Reference Types}

Reference by number: \autoref{sec:intro}.

Reference by name: ``\nameref{sec:intro}''.

Page reference: \autoref{fig:method} appears on page~\pageref{fig:method}.

Equation range: Equations~\eqref{eq:einstein}--\eqref{eq:heat}.

Multiple figures: Figures~\ref{fig:method} and~\ref{fig:results}.

% === Exercise 5: Hyperref Customization ===
% Problem: Colored links, PDF metadata, bookmarks.

\section*{Exercise 5: Hyperref Customization}

This document already loads \texttt{hyperref} with custom settings.
The configuration used is:

\begin{verbatim}
\usepackage[
  colorlinks=true,
  linkcolor=blue,
  citecolor=green,
  urlcolor=red,
  bookmarks=true,
  pdfauthor={Study Project},
  pdftitle={Cross-References Exercises},
  pdfsubject={LaTeX Exercises},
  pdfkeywords={LaTeX, cross-references, citations}
]{hyperref}
\end{verbatim}

Visit the \LaTeX{} Project: \url{https://www.latex-project.org/}

Or with custom text: \href{https://www.latex-project.org/}{\LaTeX{} Project}.

% === Exercise 6: Index ===
% Problem: Document with 20+ index entries, subentries, cross-refs, bold.

\section*{Exercise 6: Index}

\begin{verbatim}
\documentclass{article}
\usepackage{makeidx}
\makeindex
\begin{document}

LaTeX\index{LaTeX} is built on TeX\index{TeX}.
The article\index{document classes!article} class
is for short documents.
The report\index{document classes!report} class
adds chapters.
The book\index{document classes!book} class
supports front matter.
Mathematics\index{mathematics} is a key strength.
Equations\index{mathematics!equations} are numbered.
Fractions\index{mathematics!fractions} use \verb|\frac|.
Integrals\index{mathematics!integrals} use \verb|\int|.
Figures\index{floats!figures} float by default.
Tables\index{floats!tables} also float.
Bibliography\index{bibliography|textbf} management is essential.
BibTeX\index{bibliography!BibTeX} is traditional.
BibLaTeX\index{bibliography!BibLaTeX} is modern.
Hyperlinks\index{hyperlinks} make PDFs interactive.
Cross-references\index{cross-references|textbf} connect elements.
Packages\index{packages} extend functionality.
amsmath\index{packages!amsmath} improves math.
graphicx\index{packages!graphicx} includes images.
booktabs\index{packages!booktabs} formats tables.
Compilation\index{compilation} produces PDFs.
pdflatex\index{compilation!pdflatex} is standard.
See also fonts\index{fonts|see{typography}}.

\printindex
\end{document}
% Compile: pdflatex doc.tex && makeindex doc.idx && pdflatex doc.tex
\end{verbatim}

% === Exercise 7: Glossary ===
% Problem: 5 terms, 3 acronyms, usage throughout.

\section*{Exercise 7: Glossary}

\begin{verbatim}
\documentclass{article}
\usepackage[acronym]{glossaries}
\makeglossaries

\newglossaryentry{latex}{
  name=LaTeX,
  description={A document preparation system built on TeX}}
\newglossaryentry{float}{
  name=float,
  description={A container for figures or tables that LaTeX
    positions automatically}}
\newglossaryentry{preamble}{
  name=preamble,
  description={The part of a LaTeX document before
    \texttt{\textbackslash begin\{document\}}}}
\newglossaryentry{macro}{
  name=macro,
  description={A custom command defined with \texttt{\textbackslash
    newcommand}}}
\newglossaryentry{compilation}{
  name=compilation,
  description={The process of converting a .tex file into a PDF}}

\newacronym{pdf}{PDF}{Portable Document Format}
\newacronym{toc}{TOC}{Table of Contents}
\newacronym{ctan}{CTAN}{Comprehensive TeX Archive Network}

\begin{document}
\Gls{latex} produces \gls{pdf} files through \gls{compilation}.
The \gls{preamble} configures packages and \glspl{macro}.
\Glspl{float} position figures and tables automatically.
The \gls{toc} provides navigation.
Packages are available from \gls{ctan}.

\printglossaries
\end{document}
% Compile: pdflatex doc && makeglossaries doc && pdflatex doc
\end{verbatim}

% === Exercise 8: Complete Academic Paper ===
% Problem: Full paper 3+ pages.  Shown as verbatim template.

\section*{Exercise 8: Complete Academic Paper}

This exercise combines all techniques from the lesson.  The complete
solution requires external files (\texttt{.bib}, images).  The structure
is outlined below:

\begin{verbatim}
\documentclass{article}
\usepackage{amsmath,graphicx,booktabs}
\usepackage[backend=biber,style=ieee]{biblatex}
\usepackage{hyperref}
\usepackage{cleveref}
\addbibresource{refs.bib}

\title{Deep Learning for Image Classification}
\author{Jane Doe}
\date{\today}

\begin{document}
\maketitle
\begin{abstract} ... \end{abstract}
\tableofcontents

\section{Introduction}\label{sec:intro}
We study ... \parencite{lecun1998,vaswani2017}.

\section{Methods}\label{sec:methods}
\begin{equation} y = \sigma(Wx + b) \label{eq:model} \end{equation}
See \cref{fig:arch,tab:data,eq:model}.

% 3 figures (one with subfigures), 2 tables
\begin{figure}[htbp] ... \caption{...}\label{fig:arch} \end{figure}
\begin{figure}[htbp] ... \caption{...}\label{fig:curve} \end{figure}
\begin{figure}[htbp] ... \caption{...}\label{fig:loss} \end{figure}
\begin{table}[htbp]  ... \caption{...}\label{tab:data} \end{table}
\begin{table}[htbp]  ... \caption{...}\label{tab:results} \end{table}

\section{Results}\label{sec:results}
\Cref{tab:results} shows ...

\section{Conclusion}
We demonstrated ... \parencite{sutton2018,goodfellow2016,bishop2006}.

\printbibliography
\end{document}
\end{verbatim}

\end{document}
