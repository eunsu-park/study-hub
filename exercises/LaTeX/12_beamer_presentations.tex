% Exercises for Lesson 12: Beamer Presentations
% Topic: LaTeX
% Solutions to practice problems from the lesson.
% Compile: pdflatex exercises/LaTeX/12_beamer_presentations.tex

\documentclass{beamer}

\usetheme{Madrid}
\usecolortheme{beaver}

\usepackage[utf8]{inputenc}
\usepackage[T1]{fontenc}
\usepackage{amsmath, amssymb}
\usepackage{booktabs}

\setbeamertemplate{navigation symbols}{}

\title{Exercises for Lesson 12: Beamer Presentations}
\subtitle{LaTeX Study Project}
\author{Study Project}
\institute{Self-Study}
\date{}

\begin{document}

% === Exercise 1: Basic Presentation ===
% Problem: 5 slides, title, outline, 3 content, theme, list.

\begin{frame}
  \titlepage
\end{frame}

\begin{frame}{Outline}
  \tableofcontents
\end{frame}

\section{Introduction}

\begin{frame}{What is \LaTeX{}?}
  \begin{itemize}
    \item A document preparation system
    \item Built on \TeX{} by Donald Knuth
    \item The gold standard for scientific publishing
    \item Free and open source
  \end{itemize}
\end{frame}

\section{Features}

\begin{frame}{Key Features}
  \begin{enumerate}
    \item Superior mathematical typesetting
    \item Automatic numbering and cross-referencing
    \item Professional bibliography management
    \item Programmable graphics with TikZ
  \end{enumerate}
\end{frame}

\section{Conclusion}

\begin{frame}{Summary}
  \begin{block}{Key Takeaway}
    \LaTeX{} is essential for anyone writing technical or scientific
    documents.
  \end{block}
\end{frame}

% === Exercise 2: Overlays ===
% Problem: pause, \only, \uncover, <+-> itemize, overlay ranges.

\section{Overlay Demonstrations}

\begin{frame}{Using \texttt{\textbackslash pause}}
  First point is visible immediately.
  \pause

  Second point appears after one click.
  \pause

  Third point appears last.
\end{frame}

\begin{frame}{Using \texttt{\textbackslash only}}
  \only<1>{This text appears ONLY on slide 1.}
  \only<2>{This DIFFERENT text appears ONLY on slide 2.}
  \only<3>{And this text appears ONLY on slide 3.}
\end{frame}

\begin{frame}{Using \texttt{\textbackslash uncover}}
  \uncover<1->{Always visible from the start.}

  \uncover<2->{Appears from slide 2 onward.}

  \uncover<3->{Appears from slide 3 onward.}
\end{frame}

\begin{frame}{Incremental Itemize}
  \begin{itemize}[<+->]
    \item First item appears
    \item Second item appears
    \item Third item appears
    \item Fourth item appears
  \end{itemize}
\end{frame}

\begin{frame}{Overlay Ranges}
  \begin{itemize}
    \item<1-3> Visible on slides 1, 2, and 3
    \item<2-> Visible from slide 2 onward
    \item<3> Visible ONLY on slide 3
    \item<1,3> Visible on slides 1 and 3
  \end{itemize}
\end{frame}

% === Exercise 3: Blocks and Themes ===
% Problem: Regular, alert, example blocks; theorem + proof.

\section{Block Types}

\begin{frame}{Block Types}
  \begin{block}{Regular Block}
    This is a standard block with information.
  \end{block}

  \begin{alertblock}{Alert Block}
    This is an important warning or alert.
  \end{alertblock}

  \begin{exampleblock}{Example Block}
    This is an example or illustration.
  \end{exampleblock}
\end{frame}

\begin{frame}{Theorem and Proof}
  \begin{theorem}[Pythagorean Theorem]
    For a right triangle with legs $a$, $b$ and hypotenuse $c$:
    \[ a^2 + b^2 = c^2 \]
  \end{theorem}

  \begin{proof}
    Consider a square of side $a+b$ containing four copies of the
    triangle. The inner square has side $c$, giving area
    $c^2 = (a+b)^2 - 4 \cdot \frac{ab}{2} = a^2 + b^2$.
  \end{proof}
\end{frame}

% === Exercise 4: Columns Layout ===
% Problem: Equal, unequal (30/70), text+image, three columns.

\section{Column Layouts}

\begin{frame}{Two Equal Columns}
  \begin{columns}
    \column{0.5\textwidth}
    \textbf{Left Column}
    \begin{itemize}
      \item Point one
      \item Point two
    \end{itemize}

    \column{0.5\textwidth}
    \textbf{Right Column}
    \begin{itemize}
      \item Point three
      \item Point four
    \end{itemize}
  \end{columns}
\end{frame}

\begin{frame}{Unequal Columns (30/70)}
  \begin{columns}
    \column{0.3\textwidth}
    \begin{alertblock}{Key Fact}
      $E = mc^2$
    \end{alertblock}

    \column{0.7\textwidth}
    Einstein's mass-energy equivalence tells us that mass and energy are
    interchangeable.  A small amount of mass corresponds to a tremendous
    amount of energy due to the $c^2$ factor.
  \end{columns}
\end{frame}

\begin{frame}{Text and Image}
  \begin{columns}
    \column{0.5\textwidth}
    The figure on the right shows a placeholder for an image.
    In a real presentation this would contain a diagram,
    photograph, or chart.

    \column{0.5\textwidth}
    \centering
    \rule{0.9\textwidth}{4cm}\\
    {\small (Image placeholder)}
  \end{columns}
\end{frame}

\begin{frame}{Three Columns}
  \begin{columns}
    \column{0.33\textwidth}
    \begin{block}{Model A}
      Accuracy: 92\%
    \end{block}

    \column{0.33\textwidth}
    \begin{block}{Model B}
      Accuracy: 95\%
    \end{block}

    \column{0.33\textwidth}
    \begin{block}{Model C}
      Accuracy: 97\%
    \end{block}
  \end{columns}
\end{frame}

% === Exercise 5: Figures and Tables ===
% Problem: Figure slide, table slide, incremental table.

\section{Figures and Tables}

\begin{frame}{Figure Slide}
  \begin{figure}
    \centering
    \rule{0.7\textwidth}{4cm}
    \caption{Experimental results (placeholder)}
  \end{figure}
\end{frame}

\begin{frame}{Table Slide}
  \begin{table}
    \centering
    \caption{Performance metrics}
    \begin{tabular}{lcc}
      \toprule
      Model & Accuracy & F1 \\
      \midrule
      Baseline & 0.85 & 0.83 \\
      Proposed & 0.93 & 0.91 \\
      \bottomrule
    \end{tabular}
  \end{table}
\end{frame}

\begin{frame}{Incremental Table Reveal}
  \begin{table}
    \centering
    \begin{tabular}{lcc}
      \toprule
      Model & Accuracy & Speed \\
      \midrule
      \onslide<1->{CNN & 92\% & Fast \\}
      \onslide<2->{ResNet & 95\% & Medium \\}
      \onslide<3->{ViT & 97\% & Slow \\}
      \bottomrule
    \end{tabular}
  \end{table}
\end{frame}

% === Exercise 6: Custom Styling ===
% Problem: No nav symbols, custom footline, custom colors, custom block.

\section{Custom Styling}

\begin{frame}{Custom Footer Demonstration}
  This presentation has custom styling:
  \begin{itemize}
    \item Navigation symbols removed
    \item Beaver color theme applied
    \item Madrid theme with customized appearance
  \end{itemize}

  \medskip

  The footer configuration in preamble:
  \begin{verbatim}
\setbeamertemplate{navigation symbols}{}
\setbeamertemplate{footline}{
  \leavevmode\hbox{
    \begin{beamercolorbox}[wd=.33]{} Author \end{...}
    \begin{beamercolorbox}[wd=.34]{} Title \end{...}
    \begin{beamercolorbox}[wd=.33]{} \thepage \end{...}
  }
}
  \end{verbatim}
\end{frame}

% === Exercise 7: Academic Presentation (structure shown) ===

\section{Academic Structure}

\begin{frame}{Academic Presentation Structure}
  A 10-slide academic talk should have:
  \begin{enumerate}
    \item Title page
    \item Table of contents
    \item Introduction (2 slides)
    \item Methods (3 slides)
    \item Results (2 slides)
    \item Conclusion
    \item Thank you / Questions (plain frame)
  \end{enumerate}
\end{frame}

% === Exercises 8-10: Templates shown as verbatim ===

\begin{frame}[fragile]{Exercises 8--10: Notes}
  \begin{itemize}
    \item \textbf{Exercise 8} (Animations): Use
      \verb|\transfade|, \verb|\transwipe|, etc.
    \item \textbf{Exercise 9} (Speaker Notes):
      \verb|\note{Talking points here}|
    \item \textbf{Exercise 10} (Complete Presentation):
      Combine all techniques from this lesson
  \end{itemize}
\end{frame}

% Final slide
\begin{frame}[plain]
  \begin{center}
    {\Huge Thank You!}

    \bigskip

    {\Large Questions?}
  \end{center}
\end{frame}

\end{document}
